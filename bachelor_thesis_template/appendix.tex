\appendix
\renewcommand{\appendixtocname}{Appendix}
\renewcommand{\appendixpagename}{\appendixtocname}
\addappheadtotoc
\setboolean{@twoside}{false}
\appendixpage

\chapter{Lists}
\addcontentsline{toc}{section}{List of Abbreviations}
\begin{acronym}[\hspace{3cm}]
%  \acro{ac}[AC]{Acronym Without Citation}
%  \acro{ac2}[AC2]{Acronym With Citation \cite{citeKey2}}
  \acro{fol}[FOL]{First order logic}
  \acro{atp}[ATP]{Automated theorem proving}
  \acro{epr}[EPR]{Effectively propositional calculus}
  \acro{pnf}[PNF]{Prenex normal form}
  \acro{snf}[SNF]{Skolem normal form}
  \acro{cnf}[CNF]{Clausal normal form}
  \acro{crr}[CRR]{Classical Range Restricting Transformation}
  \acro{rr}[RR]{Range Restricting Transformation}
  \acro{bs}[BS]{Basic Shifting Transformation}
  \acro{pf}[PF]{Partial Flattening Transformation}
  \acro{bl}[BL]{Blocking Transformation}
\end{acronym}
\clearpage

\listoffigures
\addcontentsline{toc}{section}{List of Figures}

\listoftables
\addcontentsline{toc}{section}{List of Tables}

\lstlistoflistings
\addcontentsline{toc}{section}{List of Examples}


\chapter{Syntax of \ac{fol}}\label{chap:appendix_fol}
\subsubsection{Syntax of \ac{fol}}
The syntax of \ac{fol} consists of:
	\begin{itemize}
		\item Predicates, which is a mapping for properties in a language. Moreover, the symbols used for representing predicates are finite and specific for each problem.
		\item Terms, which consists of functions and variables as shown below:		
			\begin{itemize}
				\item Functions, which itself can be divided according to the arity of the function symbol as follows:
					\begin{itemize}
						\item if (arity == 0), then it is considered a constant
						\item if (arity $>$ 0), then it is a proper function symbol 
					\end{itemize}
				\item Variables, and they are infinite list of symbols
			\end{itemize}
		\item Special symbols
			\begin{itemize}
				\item $\bot$ which represents false
				\item $\top$ which represents true
			\end{itemize}									
	\end{itemize}


Atoms which are the basic building blocks of a formula, follow the following format:
$$ p(t1,...,tk)$$ where p is a predicate symbols, any ti is a term, and k is the arity of the predicate symbol p. A Literal is an atom or a negated atom. A formula consists of only one atom is called an Atomic formula.

Compound formulas are formed by:
	\begin{itemize}
		\item Connectives, and they are divided into:
			\begin{itemize}
				\item $\rightarrow$ : implication
				\item $\neg$ or $\sim$ : negation
				\item $\vee$ or $\vert$ : disjunction
				\item $\wedge$ or $\&$ : conjunction
				\item $\equiv$ : equivalence
			\end{itemize}				
		\item Quantifiers, and they are the following two:
			\begin{itemize}
				\item $\forall$ which is the universal quantifier
				\item $\exists$ which is the existential quantifier
			\end{itemize}
	\end{itemize}


A Clause is a disjunction of Literals. A ground clause is a clause having no variables. A positive clause is a clause who has no negated atoms. While a negative clause is a clause who contains only negated atoms. A mixed clause is a clause who consists of both atoms and negated atoms. A unit clause is a clause containing one Literal.


%TODO : add the commented chapter back

\begin{comment}
\chapter{Forms of first order logic formulas}
Different forms of first order logic formulas.

\chapter{Algorithms}
Different Algorithms used.
\end{comment}