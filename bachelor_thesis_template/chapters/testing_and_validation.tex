\chapter{Testing, Validation and Evaluation}\label{chap:test_and_val}
Here in this chapter, we explain what was done for validating the output and Evaluating the program. So we divide this chapter into two sections:
\begin{itemize}
	\item Validating and Evaluating the Simplified Transformations.
	\item Validating and Evaluating the Implementation of Bachmair and Ganzinger Model Construction Technique.
\end{itemize} 

\section{Simplified Transformations}

\subsection{Evaluating Memory Efficiency}\label{sub:val_tran_mem}
The first part of the implementation which is related to applying the transformations to the the clause set introduces no memory leaks to the whole program.
%TODO add number of problems checked upon


Tools to check this:
\begin{enumerate}
	\item Script implemented in E for giving a summary on the allocated and de-allocated memory structures, and the results showed that they are equal.
		\begin{lstlisting}
# -------------------------------------------------
# Total SizeMalloc()ed memory: 68536168 Bytes (131507 requests)
# Total SizeFree()ed   memory: 68536168 Bytes (131507 requests)
# New requests: 214 (197 by SecureMalloc(), 17 by SecureRealloc())
# Total SecureMalloc()ed memory: 277647 Bytes
# Returned: 214 (214 by FREE(), 0 by SecureRealloc())
# SecureRealloc(ptr): 19 (17 Allocs, 0 Frees, 2 Reallocs)
# -------------------------------------------------
		\end{lstlisting}
	\item Tool named 'valgrind' which also showed the same results as the above script.
\end{enumerate}


\subsection{Accuracy of the transformations}
In all tested problems, The results of the Transformations were the same as an implemented program called Yarralumla \url{http://users.cecs.anu.edu.au/~baumgart/systems/yarralumla/}, implemented by Peter Baumgartner, one of the authors of \cite{BMUG06}. So we believe it is correct.

In order to check the correctness of the transformations, we implemented a c program "tptp2tme" that takes a tptp problem and then transform it to tme syntax, which is extended prolog language and the one used in the yarralumla program. Then we applied yarralumla to the outputed problem and Our transformations to the original tptp problem. Afterwards, we compared the output of both implementations. In all tested problems they were exactly the same. So we believe in this case our implementation is correct.


\section{Bachmair and Ganzinger Model Construction Technique}
\subsection{Evaluating Memory Efficiency}
The same tools mentioned for The Transformations in \ref{sub:val_tran_mem} part were used. And the same results were achieved for the Implemented Model Construction Technique.



\subsection{Complexity of the algorithm}



\subsection{Accuracy of Implementation}
To Be added later 