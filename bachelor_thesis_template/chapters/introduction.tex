\chapter{Introduction}\label{chap:intro}

	\paragraph{    Constructing Models} has a huge importance in many fields specially in debugging tasks. It helps in modelling and highlighting the existence of bugs. And because of this it is used extensively in the field of Software and Hardware verification, also in analyzing and verifying Theorems.

	\paragraph{    So a concrete example} to show its importance is the verification of Timsort, which was explained in details here \cite{TIMSORT}. Timsort is a hybrid sorting algorithm that was developed in 2002. It combines merge sort and insertion sort. And it was developed in the beginning to be used in python, but later on it was added to java. And today in Open JDK, Sun's JDK, and Android SDK it is the default sorting algorithm since it showed a great performance on real data. That resulted in using it in billions of devices because of the popularity of those platforms. In 2015, a formal verification for Timsort was tried to be done by a team using KeY, a verification tool for java programs that could be found here \url{http://www.key-project.org/}. And their analysis showed that the TimSort algorithm was broken and they found a bug and they corrected it, then after that they were able to formally verify the correction of the algorithm. And all the happened by the help of KeY. So it really beneficial to have models.

	\paragraph{    Having that importance in mind,} we needed in this project to have an explicit model given to user when running problems on the theorem prover E, specifically for a sub-class of \ac{fol} named \ac{epr}. And in order to achieve that goal, Transformations have been applied to the problem specification to transform it to a certain form, namely range restricted form. Afterwards, Bachmair and Ganzinger Model Construction Technique is applied to extract the explicit model from the saturated set of the problem specification. 


	\paragraph{    So a discussion of the work done} will be given in the following order. We will have a background on the topic in chapter \ref{chap:background}. Then in chapter \ref{chap:meth_and_impl} we will discuss the methodology followed and the implementation. Afterwards in chapter \ref{chap:test_and_val} we will explain the procedures followed to test the accuracy and efficiency of the implemented techniques. Moreover, A discussion of the results and related works will be found in chapter \ref{chap:res_and_lit}. Then a conclusion will be given in chapter \ref{chap:concl}. And last but not least a discussion for related future work will be in chapter \ref{chap:todo} 
