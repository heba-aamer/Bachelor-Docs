\chapter{Background}\label{chap:background}


This chapter is devoted for introducing and familiarizing the reader to the theoretical concepts behind this project, and give the reader a background on the theorem prover E as well. So it will include the following sections:
	\begin{itemize}
		\item A background on \ac{fol} in section %TODO add a reference
		\item A background on \ac{epr} in section %TODO add a reference
		\item A background on \ac{atp} in section %TODO add a reference
		\item A background on E in section %TODO add a reference
		\item A background on Problem set in section %TODO add a reference
	\end{itemize}


\section{Background on first order logic topics} \label{sec:c2s1}

\ac{fol}, which is also known as first order predicate logic, is an expressive logic that allows us to formulate most of our spoken language sentences in a defined way such that it could be further handled with rules such as simplification, inference rules, etc.\\*
We could represent formulas in \ac{fol} in so many forms. So \ref{sub:c2s1s1} will be devoted to that part of background.\newline
Moreover, in general the universe of \ac{fol} is infinite because of the existence of the quantifiers and function terms. So in \ref{sub:c2s1s2} some topics related to that property will be mentioned including Herbrand Universe.\newline    
Also this project is concerned to a specific class of \ac{fol} its calculus named \ac{epr}, and this will be discussed in \ref{sub:c2s1s4}.

\subsection{Different forms of \ac{fol}}\label{sub:c2s1s1}
\ac{fol} can be represented in different forms. Even some algorithms need to work with specific one.
Moreover, there are some procedures that could transform a formula from one form to another.
Here in this part, we will discuss the most important forms and the relevant ones to the project.
\newline
1- general form\newline 
2- clausal normal form\newline
3- prenex normal form\newline
4- skolem normal form

\subsubsection{General Form}
General form

\subsubsection{Clausal normal form}
\ac{cnf}

\subsubsection{Prenex normal form}
\ac{pnf}

\subsubsection{Skolem normal form}
\ac{snf}

\subsection{The universe of \ac{fol}}\label{sub:c2s1s2}
Discuss the infinity of the universe in general.\newline
And then mention the Herbrand Universe. 

\subsection{Skolemization}\label{sub:c2s1s3}
Skolemization is the step that transforms \ac{pnf} formulas into \ac{snf}.
In which all the existentially quantified variables are removed and replaced by some function terms and its arguments are all the universally quantified variables appeared before the one in concern.\\
Example for a formula in \ac{pnf}:\newline
\begin{displaymath}
\exists W \forall X \forall Y \exists Z  \left(P(a,W,X,Y,Z)\right).  
\end{displaymath}\newline
After it transformed into \ac{snf}:\newline
\begin{displaymath}
\forall X \forall Y  \left(P(a,b,X,Y,f(X,Y))\right).  
\end{displaymath}

%TODO add link to the algorithm of the transformation in the appendix. 

\subsection{\ac{epr}}\label{sub:c2s1s4}
\ac{epr} or sometimes known as Bernays-Schoenfinkel class
is a class of first order logic in which all of its formulas follow the following format:\\*
$\exists *$ $\forall *$ F, where F is the formula.
Moreover, F has no proper functions symbols (all functions present are nullary ones "constants").\newline
Example for a formula in \ac{epr}:\newline
\begin{displaymath}
\exists X \exists Y \forall Z  \left(P(a,X,Y,Z)\right).
\end{displaymath}
\\
Keeping \ref{sub:c2s1s3} in mind, this will make every skolemized formula that originally was in \ac{epr} format have no proper function symbols.
After transforming the above equation, it will be: \newline
\begin{displaymath}
\forall Z \left( P(a,b,c,Z) \right).
\end{displaymath}



   



\section{\acf{atp}}



\section{E's implementation and state} \label{sec:c2s2}
E is a saturation-based theorem prover. It is known to be a fast one because of the unique and various heuristics implemented in it.  



\section{Problem Set}
TPTP problems are first order problems, that are considered benchmarks to measure the performance of the different automated theorem provers. A description for TPTP could be found in \cite{TPTP09}, and the problem set could be downloaded from \url{http://www.cs.miami.edu/~tptp/}. TPTP problem set consists of different categories of problems, or divisions in other words, such as : PUZ, NLP, GRP. Those categories depends on the problems scientific meaning, e.g., NLP represents Natural language processing problems.


TPTP has its own language; TPTP for problems and TSTP for solutions. TPTP language could be written in two formats. The first of them is FOF, which is first order form. While the second form is CNF, which is clause normal form. Having a unified language for problems and solutions had a great impact on the field of \ac{atp}. And this allows the integration of different \ac{atp} systems ,i.e., Automated theorem provers, since they have a specific language that could communicate in.  


An example for a problem, i.e., GRP004-1 problem, from the TPTP problem set is given below in CNF format:

	\begin{lstlisting}[caption=GRP004-1.p problem,basicstyle=\footnotesize,breaklines=true,frame=single]
%--------------------------------------------------------------------------
% File     : GRP004-1 : TPTP v6.1.0. Released v1.0.0.
% Domain   : Group Theory
% Problem  : Left inverse and identity => Right inverse exists
% Version  : [Cha70] axioms : Incomplete.
% English  : In a group with left inverses and left identity every element
%            has a right inverse.

% Refs     : [Luc68] Luckham (1968), Some Tree-paring Strategies for Theore
%          : [Cha70] Chang (1970), The Unit Proof and the Input Proof in Th
%          : [CL73]  Chang & Lee (1973), Symbolic Logic and Mechanical Theo
% Source   : [Cha70]
% Names    : Example 3 [Luc68]
%          : Example 4 [Cha70]
%          : Example 4 [CL73]
%          : EX4 [SPRFN]

% Status   : Unsatisfiable
% Rating   : 0.00 v5.4.0, 0.11 v5.3.0, 0.10 v5.2.0, 0.00 v2.1.0, 0.00 v2.0.0
% Syntax   : Number of clauses     :    5 (   0 non-Horn;   3 unit;   3 RR)
%            Number of atoms       :   11 (   0 equality)
%            Maximal clause size   :    4 (   2 average)
%            Number of predicates  :    1 (   0 propositional; 3-3 arity)
%            Number of functors    :    3 (   2 constant; 0-1 arity)
%            Number of variables   :   15 (   1 singleton)
%            Maximal term depth    :    2 (   1 average)
% SPC      : CNF_UNS_RFO_NEQ_HRN

% Comments : [Luc68] is actually the right to left version.
%--------------------------------------------------------------------------
cnf(left_inverse,axiom,
    ( product(inverse(X),X,identity) )).
cnf(left_identity,axiom,
    ( product(identity,X,X) )).
cnf(associativity1,axiom,
    ( ~ product(X,Y,U)
    | ~ product(Y,Z,V)
    | ~ product(U,Z,W)
    | product(X,V,W) )).
cnf(associativity2,axiom,
    ( ~ product(X,Y,U)
    | ~ product(Y,Z,V)
    | ~ product(X,V,W)
    | product(U,Z,W) )).
cnf(prove_there_is_a_right_inverse,negated_conjecture,
    ( ~ product(a,X,identity) )).
%--------------------------------------------------------------------------
	\end{lstlisting}