\chapter{Conclusion}\label{chap:concl}


So we can conclude from the results of testing that the saturated output did not have a significant difference after applying the range restricting transformations for the problem specification. Most of the solved problems were almost range restricted problems from the beginning. So no clear benefit from the transformations were gained.


However, Something that was not tested thoroughly was the effect of termination in general for \ac{epr} problems since in testing a time limit of three minutes were given to halt if it did not achieve results before it. As it was mentioned before that resolution based reasoning systems do not guarantee termination. However, The transformations guarantee termination for \ac{epr}. So this needs to be checked out without a time limit.


At the end, Minimal explicit model was given after applying the Bachmair and Ganzinger Model Construction Technique for the saturated set. Another addition was the extension of Bachmair and Ganzinger Model Construction Technique to handle equality in \ac{epr} formulae.
