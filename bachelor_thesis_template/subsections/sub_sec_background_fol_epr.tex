\subsection{\ac{epr}}\label{sub:c2s1s4}
\ac{epr} or sometimes known as Bernays-Schoenfinkel class
is a class of first order logic in which all of its formulas follow the following format:\\*
$\exists *$ $\forall *$ F, where F is the formula.
Moreover, F has no proper functions symbols (all functions present are nullary ones "constants").\newline
Example for a formula in \ac{epr}:\newline
\begin{displaymath}
\exists X \exists Y \forall Z  \left(P(a,X,Y,Z)\right).
\end{displaymath}
\\
Keeping \ref{sub:c2s1s3} in mind, this will make every skolemized formula that originally was in \ac{epr} format have no proper function symbols.
After transforming the above equation, it will be: \newline
\begin{displaymath}
\forall Z \left( P(a,b,c,Z) \right).
\end{displaymath}
