\subsection{Original transformations}\label{sub:c3s1s1}
The original procedures discussed in \cite{BMUG06} generally work for all sub-classes of \ac{fol}.
Those procedures should be applied to a given set of axioms in a specific form "implication form" which is explained here -- , and here -- to know how to transform to that form.
\\
Transformations are series of procedures, mainly about changing the clauses to certain form named range restricted form. 
In general the transfromations deal with clauses in the implication from.
Range restricted clauses: are the clauses in which all the variables that appear in the succeedent must exist in the anticedent as well.
Transformations also add a domain predicate that will help in finding the interpretations.
\\
There are 3 types of transformations:
\\
First 2 transformations: "Range restricting transformations" 'crr' and a variant of it 'rr'
mainly responsible for changing the clauses to the range restricted form.
The difference is that 'rr' adds elements to the domain when it needs them, while 'crr' enumerate the Herbrand Universe.
\\
Second 2 transfromations: "Shifting transformations" 'bs' and 'pf'
mainly for preventing non-termination and generating unpleasant clauses from steps in 'rr' 
\\
Last transfromation: "Blocking transformation" 'bl'
mainly for detecting periodictiy, so it defines a new predicate sub/2 that represents a subterm relation
\\
Order of applying those transformations is shifting -> range restricting -> blocking
