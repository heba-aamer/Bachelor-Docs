\subsection{Skolemization}\label{sub:c2s1s3}
Skolemization is the step that transforms \ac{pnf} formulas into \ac{snf}.
In which all the existentially quantified variables are removed and replaced by some function terms and its arguments are all the universally quantified variables appeared before the one in concern.\\
Example for a formula in \ac{pnf}:\newline
\begin{displaymath}
\exists W \forall X \forall Y \exists Z  \left(P(a,W,X,Y,Z)\right).  
\end{displaymath}\newline
After it transformed into \ac{snf}:\newline
\begin{displaymath}
\forall X \forall Y  \left(P(a,b,X,Y,f(X,Y))\right).  
\end{displaymath}

%TODO add link to the algorithm of the transformation in the appendix. 
