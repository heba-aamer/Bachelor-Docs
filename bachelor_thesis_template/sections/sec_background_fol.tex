\section{\acf{fol}} \label{sec:c2s1}

\ac{fol}, which is also known as first order predicate logic, is an expressive logic that allows us to formulate and encode most of our spoken language sentences in a defined way such that it could be further handled with rules such as simplification, inference rules. That allows automating the reasoning for \ac{fol} problems.


In order to have a general background on the topics of \ac{fol}, we will devote this section to that mission. So we are going to discuss the following points:
	
	\begin{itemize}
		\item Summary over \ac{fol} and Example on using it
		\item Decidability of \ac{fol}
		\item Herbrand Universe
	\end{itemize}	    

%TODO : remove comment later
\begin{comment}
	We could represent formulas in \ac{fol} in so many forms. So \ref{sub:c2s1s1} will be devoted to that part of background.
\end{comment}



\subsection{\ac{fol} overview and Example on it}

\subsubsection{Overview on \ac{fol}}
Quantifiers, variables, and functions are what signifies \ac{fol} over propositional logic. Being able to formulate "some", "all" is what added a lot to the expressiveness of \ac{fol}. So a review on the syntax and the structure of \ac{fol} if needed will be found in the appendix here \ref{chap:appendix_fol}.


\subsubsection{Example on \ac{fol}}
A very famous example on \ac{fol} is the following:
\begin{lstlisting}[caption=Example on \ac{fol},mathescape,breaklines=true,frame=single]

 We want to formulate the following three sentences in $\ac{fol}$:

  (1) All Humans are mortals.
  (2) Socrates is a human.
  (3) Therefore, Socrates is mortal.
  
 So in $\ac{fol}$ syntax they are:

  (1) $\forall X (Human(X) \rightarrow Mortal(X))$
  (2) $Human(socrates)$
  (3) $Mortal(socrates)$
  
 Where $Human$ and $Mortal$ are predicates here since they represent property over the elements of the domain. While $socrates$ is a constant or in other words a function symbol with arity 0. 

\end{lstlisting}


\subsection{Decidability of \ac{fol}}
\paragraph{}
The Problem of proving validity, or in other words unsatisfiability, of a formula in \acf{fol} had a lot of attention and experiments in the beginning. A lot of trials were made to prove that it is decidable as mentioned in \cite{SL_14}. But those trials were not successful. Later on, in the same year both \cite{DEC_TUR, DEC_CHURCH} proved that this problem in general is un-decidable.


Some algorithms were developed such that if the formula is unsatisfiable it will give a refutation, or proof in simpler words, however if it was not unsatisfiable then the algorithm may halt/terminate and give the correct result and may not halt/terminate, so this problem is semi-decidable. An example for such procedure is resolution, with some refinements, in which its basic idea was first developed in \cite{RES_65}. So \acf{fol} is semi-decidable logic.

\subsection{Universe of \ac{fol}}


%TODO : remove comment later
\begin{comment}
	\subsection{Different forms of \ac{fol}}\label{sub:c2s1s1}
\ac{fol} can be represented in different forms. Even some algorithms need to work with specific one.
Moreover, there are some procedures that could transform a formula from one form to another.
Here in this part, we will discuss the most important forms and the relevant ones to the project.
\newline
1- general form\newline 
2- clausal normal form\newline
3- prenex normal form\newline
4- skolem normal form

\subsubsection{General Form}
General form

\subsubsection{Clausal normal form}
\ac{cnf}

\subsubsection{Prenex normal form}
\ac{pnf}

\subsubsection{Skolem normal form}
\ac{snf}


	\subsection{The universe of \ac{fol}}\label{sub:c2s1s2}
Discuss the infinity of the universe in general.\newline
And then mention the Herbrand Universe. 

	\subsection{Skolemization}\label{sub:c2s1s3}
Skolemization is the step that transforms \ac{pnf} formulas into \ac{snf}.
In which all the existentially quantified variables are removed and replaced by some function terms and its arguments are all the universally quantified variables appeared before the one in concern.\\
Example for a formula in \ac{pnf}:\newline
\begin{displaymath}
\exists W \forall X \forall Y \exists Z  \left(P(a,W,X,Y,Z)\right).  
\end{displaymath}\newline
After it transformed into \ac{snf}:\newline
\begin{displaymath}
\forall X \forall Y  \left(P(a,b,X,Y,f(X,Y))\right).  
\end{displaymath}

%TODO add link to the algorithm of the transformation in the appendix. 


	\subsection{\ac{epr}}\label{sub:c2s1s4}
\ac{epr} or sometimes known as Bernays-Schoenfinkel class
is a class of first order logic in which all of its formulas follow the following format:\\*
$\exists *$ $\forall *$ F, where F is the formula.
Moreover, F has no proper functions symbols (all functions present are nullary ones "constants").\newline
Example for a formula in \ac{epr}:\newline
\begin{displaymath}
\exists X \exists Y \forall Z  \left(P(a,X,Y,Z)\right).
\end{displaymath}
\\
Keeping \ref{sub:c2s1s3} in mind, this will make every skolemized formula that originally was in \ac{epr} format have no proper function symbols.
After transforming the above equation, it will be: \newline
\begin{displaymath}
\forall Z \left( P(a,b,c,Z) \right).
\end{displaymath}

\end{comment}


   