\section{\acf{fol}} \label{sec:c2s1}

\ac{fol}, which is also known as first order predicate logic, is an expressive logic that allows us to formulate and encode most of our spoken language sentences in a defined way such that it could be further handled with rules such as simplification, inference rules. That allows automating the reasoning for \ac{fol} problems.


In order to have a general background on the topics of \ac{fol}, we will devote this section to that mission. So we are going to discuss the following points:
	
	\begin{itemize}
		\item Summary over \ac{fol} and Example on using it
		\item Decidability of \ac{fol}
		\item Herbrand Universe
		\item Equality
	\end{itemize}	    

%TODO : remove comment later
\begin{comment}
	We could represent formulas in \ac{fol} in so many forms. So \ref{sub:c2s1s1} will be devoted to that part of background.
\end{comment}



\subsection{\ac{fol} overview and Example on it}

\subsubsection{Overview on \ac{fol}}
Quantifiers, variables, and functions are what signifies \ac{fol} over propositional logic. Having functions and being able to formulate "some", "all" is what added a lot to the expressiveness of \ac{fol}. So a review on the syntax and the structure of \ac{fol}, if needed, will be found here in appendix \ref{chap:appendix_fol}.


\subsubsection{Example on \ac{fol}}
A very famous example on \ac{fol} is the following:
\begin{lstlisting}[caption=Example on \ac{fol},mathescape,breaklines=true,frame=single]

 We want to formulate the following three sentences in $\ac{fol}$:

  (1) All Humans are mortals.
  (2) Socrates is a human.
  (3) Therefore, Socrates is mortal.
  
 So in $\ac{fol}$ syntax they are:

  (1) $\forall X (Human(X) \rightarrow Mortal(X))$
  (2) $Human(socrates)$
  (3) $Mortal(socrates)$
  
 Where $Human$ and $Mortal$ are predicates here since they represent properties over the elements of the domain. While $socrates$ is a function symbol with arity 0 or in other words a constant. 

\end{lstlisting}



\subsection{Decidability of \ac{fol}}
\paragraph{}
The Problem of proving validity, or in other words unsatisfiability, of a formula in \acf{fol} had a lot of attention and experiments in the beginning. A lot of trials were made to prove that it is decidable as mentioned in \cite{SL_14}. But those trials were not successful. Later on, in the same year both \cite{DEC_TUR, DEC_CHURCH} proved that this problem in general is un-decidable.


Some algorithms were developed such that if the formula is unsatisfiable it will give a refutation, or proof in simpler words, however if it was not unsatisfiable then the algorithm may halt/terminate and give the correct result or may not halt/terminate at all, so this problem is semi-decidable. An example for such procedure is resolution, with some refinements, in which its basic idea was first developed in \cite{RES_65}. So \acf{fol} is a semi-decidable logic.



\subsection{Universe of \ac{fol}}
\paragraph{}
Herbrand was a french mathematician who died at the age of 23 in 1931. Herbrand had numerous contributions in the field of logic. One of his major contributions is devising a procedure for generating the Universe of set of first order formulas; where the universe is the set of constants and grounded function symbols that represent terms in predicates. The procedure is defined recursively. So in most of the cases the Herbrand Universe of any set of first order formulas is infinite.


\subsection{Equality}
The Equality relation adds a lot of expressiveness and it is so intuitive to have it in logic. Adding equality to \ac{fol} is very beneficial such that it eases encoding problems to \ac{fol}, however it adds a burden for how it should be handled. So \ac{fol} with equality adds a distinguished predicate symbol $=$ or $\approx$ that represents equality to the set of predicate symbols. Moreover, it adds to the problem specification some axioms that helps in dealing with equality such as: reflexivity and symmetry.


In general, Equations are of the form: $t1$ $\approx$ $t2$, Where each of the t1 and t2 is a term.


%TODO : remove comments
\begin{comment}
One representation of Herbrand Theorem states that:
"Let F be a formula in CNF. The formula F is unsatisfiable iff there is an unsatisfiable set G containing only ground instances of clauses in F."


So this theorem reduces any \ac{fol} problem to an equivalent propositional problem. And this a very important reduction.
\end{comment}




\section{\acf{epr}}\label{sec:c2s2}
\paragraph{} 
This section is discussing important points related to \ac{epr}, so the following points will be covered:

	\begin{itemize}
		\item subsection \ref{sub:c2s2s1} will explain what \ac{epr} is.
		\item Example on \ac{epr} will be given in subsection \ref{sub:c2s2s2}.
		\item Universe and Decidability of \ac{epr} will discussed in subsection \ref{sub:c2s2s3}.
		\item And last but not least in subsection \ref{sub:c2s2s4} we will talk about Equality in \ac{epr}. 
	\end{itemize}


	\subsection{What is \ac{epr}}\label{sub:c2s2s1}
\acf{epr} or sometimes known as Bernays-Sch\"onfinkel class is a fragment of \acf{fol} in which all of its formulas follow the following format:

\begin{lstlisting}[caption=Format of \ac{epr} formula,label={lst:epr_format},breaklines=true,mathescape,frame=single]
                          $\exists *$ $\forall *$ $F$
  where $F$ is the formula.
  Moreover, $F$ has no proper functions symbols (all functions present are nullary ones "constants").
\end{lstlisting}

\ac{epr} fragment has a huge number of applications, e.g. Planning and Verifcation. 


\subsection{Example on \ac{epr}}\label{sub:c2s2s2}
Example for a formula in \ac{epr}:

\begin{lstlisting}[caption=Example for an \ac{epr} formula,breaklines=true,escapeinside={(*}{*)},frame=single]
(* \begin{displaymath}
		\exists X \exists Y \forall Z  \left(P(a,X,Y,Z)\right).
	\end{displaymath} *)
\end{lstlisting}


Keeping \ref{lst:epr_format} in mind, this will make every skolemized formula that originally was in \ac{epr} format have no proper function symbols. After transforming the above equation, it will be:

\begin{lstlisting}[caption=Example for a skolemized \ac{epr} formula,breaklines=true,escapeinside={(*}{*)},frame=single]
(* \begin{displaymath}
	\forall Z \left( P(a,b,c,Z) \right).
	\end{displaymath} *)
\end{lstlisting}



\subsection{Decidability and Universe of \ac{epr}}\label{sub:c2s2s3}
Since the domain of \ac{epr} formulas contains only constants, and the number of constants in a formula is finite. So Herbrand universe of \ac{epr} formulas is finite. From that we could prove that \ac{epr} fragment is a decidable fragment in \ac{fol}, because of the finite number of ground interpretations it can have.


\subsection{Equality in \ac{epr}}\label{sub:c2s2s4}
As we have mentioned before that \ac{epr} formulas will never contain proper function symbols. So terms will be one of two options:
\begin{itemize}
	\item constant
	\item variable
\end{itemize}


And since Equality may be added to \ac{fol} as discussed in subsection \ref{sub:c2s2s3}, then we should consider the case if the formulas are from the \ac{epr} fragment. Equations are in the following form $t1$ $\approx$ $t2$, where t1 and t2 are terms. And since terms in \ac{epr} are only variables and constants. So the ground case of the Equation will contain only of constants. This result is important and we will refer to it later in chapter \ref{chap:meth_and_impl}.
 





%TODO : remove comment later
\begin{comment}
	\subsection{Different forms of \ac{fol}}\label{sub:c2s1s1}
\ac{fol} can be represented in different forms. Even some algorithms need to work with specific one.
Moreover, there are some procedures that could transform a formula from one form to another.
Here in this part, we will discuss the most important forms and the relevant ones to the project.
\newline
1- general form\newline 
2- clausal normal form\newline
3- prenex normal form\newline
4- skolem normal form

\subsubsection{General Form}
General form

\subsubsection{Clausal normal form}
\ac{cnf}

\subsubsection{Prenex normal form}
\ac{pnf}

\subsubsection{Skolem normal form}
\ac{snf}


	\subsection{The universe of \ac{fol}}\label{sub:c2s1s2}
Discuss the infinity of the universe in general.\newline
And then mention the Herbrand Universe. 

	\subsection{Skolemization}\label{sub:c2s1s3}
Skolemization is the step that transforms \ac{pnf} formulas into \ac{snf}.
In which all the existentially quantified variables are removed and replaced by some function terms and its arguments are all the universally quantified variables appeared before the one in concern.\\
Example for a formula in \ac{pnf}:\newline
\begin{displaymath}
\exists W \forall X \forall Y \exists Z  \left(P(a,W,X,Y,Z)\right).  
\end{displaymath}\newline
After it transformed into \ac{snf}:\newline
\begin{displaymath}
\forall X \forall Y  \left(P(a,b,X,Y,f(X,Y))\right).  
\end{displaymath}

%TODO add link to the algorithm of the transformation in the appendix. 


	\subsection{What is \ac{epr}}\label{sub:c2s2s1}
\acf{epr} or sometimes known as Bernays-Sch\"onfinkel class is a fragment of \acf{fol} in which all of its formulas follow the following format:

\begin{lstlisting}[caption=Format of \ac{epr} formula,label={lst:epr_format},breaklines=true,mathescape,frame=single]
                          $\exists *$ $\forall *$ $F$
  where $F$ is the formula.
  Moreover, $F$ has no proper functions symbols (all functions present are nullary ones "constants").
\end{lstlisting}

\ac{epr} fragment has a huge number of applications, e.g. Planning and Verifcation. 


\subsection{Example on \ac{epr}}\label{sub:c2s2s2}
Example for a formula in \ac{epr}:

\begin{lstlisting}[caption=Example for an \ac{epr} formula,breaklines=true,escapeinside={(*}{*)},frame=single]
(* \begin{displaymath}
		\exists X \exists Y \forall Z  \left(P(a,X,Y,Z)\right).
	\end{displaymath} *)
\end{lstlisting}


Keeping \ref{lst:epr_format} in mind, this will make every skolemized formula that originally was in \ac{epr} format have no proper function symbols. After transforming the above equation, it will be:

\begin{lstlisting}[caption=Example for a skolemized \ac{epr} formula,breaklines=true,escapeinside={(*}{*)},frame=single]
(* \begin{displaymath}
	\forall Z \left( P(a,b,c,Z) \right).
	\end{displaymath} *)
\end{lstlisting}



\subsection{Decidability and Universe of \ac{epr}}\label{sub:c2s2s3}
Since the domain of \ac{epr} formulas contains only constants, and the number of constants in a formula is finite. So Herbrand universe of \ac{epr} formulas is finite. From that we could prove that \ac{epr} fragment is a decidable fragment in \ac{fol}, because of the finite number of ground interpretations it can have.


\subsection{Equality in \ac{epr}}\label{sub:c2s2s4}
As we have mentioned before that \ac{epr} formulas will never contain proper function symbols. So terms will be one of two options:
\begin{itemize}
	\item constant
	\item variable
\end{itemize}


And since Equality may be added to \ac{fol} as discussed in subsection \ref{sub:c2s2s3}, then we should consider the case if the formulas are from the \ac{epr} fragment. Equations are in the following form $t1$ $\approx$ $t2$, where t1 and t2 are terms. And since terms in \ac{epr} are only variables and constants. So the ground case of the Equation will contain only of constants. This result is important and we will refer to it later in chapter \ref{chap:meth_and_impl}.
 

\end{comment}


   