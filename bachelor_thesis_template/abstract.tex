\chapter*{Abstract}
% \addcontentsline{toc}{chapter}{Abstract}
\label{chap:abstract}

%TODO : remove comments

\begin{comment}
In the past few decades the field of automated theorem proving (ATP) has been flourishing and improving a lot by the enormous amount of research devoted to it. That interest came from its importance as well as its various uses in different fields such as mathematical reasoning.



ATP comes along with another process which is model generation/computation/construction from a certain (counter) satisfiable specification/problem. Model generation has usages that ATP alone wont have the effect that it has with it, and that could be noticed in Software/Hardware verification, debugging various systems.



Here in this project we added a model generation technique for a subclass in First Order Logic named Effectively Propositional Calculus in an existing theorem prover "E" where we transform the axioms of the specification into a certain form called range restricted form, and then after reaching saturation, we apply Bachmair and Ganzinger model construction technique to get the model.



Automated theorem proving has applications in mathematics, verification, common-sense reasoning, and many other domains. It can demonstrate the compliance of a system with certain requirements. However, it is often just as important to show that a desired property is not met. This can be done by constructing a counter-model, or, in simpler words, a counter-example. In this talk we describe the implementation of techniques that enable the theorem prover E to find such counter-examples for effectively propositional proof problems, and to give an explicit counter-models to the user.
\end{comment}

\paragraph{}
\textbf{Keywords:} Automated Theorem Proving, Model Construction, Effectively propositional logic, Range Restricting Transformations, Theorem Prover E, Bachmair and Ganzinger Model Construction Technique.  